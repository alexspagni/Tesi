\chapter*{Introduzione} % \chapter* -> the introduction isn't the Chapter 1, it's not a numbered chapter
\addcontentsline{toc}{chapter}{Introduzione} %this line enable the introduction to be listed in the Table Of Contents even if it's not a numbered chapter (see above)
\markboth{}{}
Il mondo mobile fa sempre pi\`u parte della nostra vita e ogni giorno vengono inventate sempre nuove applicazioni che permettono di semplificarla. Esistono app
per poter ascoltare la musica, altre per poter eseguire delle transazioni bancarie ed altre ancora destinate alla domotica, per fare alcuni esempi.
Per questi motivi \`e stato particolarmente interessante svolgere il tirocinio presso l'azienda Novalab SRL con sede a Reggio nell'Emilia.

L'azienda nasce nel 2017 e gli ambiti di cui si occupa sono:
\begin{itemize}
    \item Lo sviluppo Mobile
    \item Lo sviluppo back-end 
    \item Progetti su Blockchain
\end{itemize}
La maggior parte delle applicazioni sono state sviluppate attraverso l'uso del framework React Native: il che significa che sono multipiattaforma; mentre altre sono state sviluppate usando linguaggio nativo come Kotlin e Swift. \\
Per quanto riguarda lo sviluppo back-end, come web-services necessari al funzionamento dell'applicazione, viene utilizzato Node.js e quindi il linguaggio JavaScript.
Dal 2017 l'azienda ha sviluppato applicazioni per diverse compagnie, tra cui: Roadhose, Casa.it, Cedacri e MoneyFarm. In particolare per Cedacri \`e stata realizzata 
un'applicazione di Home Banking che va a servire diverse banche utilizzando un'unica codebase.

L'esperienza di tirocinio si \`e concentrata sulla comprensione e sullo sviluppo di un'applicazione mobile. Affinch\`e fosse possibile raggiungere il maggior numero di piattaforme, 
l'applicativo \`e stato realizzato usando come framework React Native.

Il seguente documento di tesi \`e strutturato secondo tale schema:
\begin{itemize}
    \item Capitolo uno: vengono descritte le varie tecnologie e linguaggi utilizzati per lo sviluppo dell'applicazione. Inoltre viene fatto un confronto fra lo sviluppo nativo e quello multipiattaforma descrivendo quando conviene seguire il primo rispetto al secondo e viceversa.
    \item Capitolo due: vengono descritti i requisiti funzionali e non funzionali necessari per lo sviluppo dell'applicazione. Sono poi illustrati i principali useCase in modo da spiegare come e perch\`e vengono eseguiti certi servizi.
    \item Capitolo tre: attraverso l'uso di frammenti di codice vengono descritte le librerie utilizzate per la realizzazione dell'applicazione e le principali funzioni all'interno di essa.   
    \item Capitolo quattro: viene illustrato il funzionamento dell'applicazione e i risultati ottenuti sulle diverse piattaforme su cui \`e stata installata, Android e iOS, descrivendo le principali differenze osservate.
\end{itemize}

